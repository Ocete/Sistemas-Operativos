\documentclass[11pt,a4paper]{article}

% Packages
\usepackage[utf8]{inputenc}
\usepackage[spanish, es-tabla]{babel}
\usepackage{caption}
\usepackage{listings}
\usepackage{adjustbox}
\usepackage{enumitem}
\usepackage{boldline}
\usepackage{amssymb, amsmath}
\usepackage[margin=1in]{geometry}
\usepackage{xcolor}
\usepackage{soul}

% Meta
\title{Título}
\author{José Antonio Álvarez Ocete}
\date{\today}

% Custom
\providecommand{\abs}[1]{\lvert#1\rvert}
\setlength\parindent{0pt}
\definecolor{Light}{gray}{.90}
\newcommand\ddfrac[2]{\frac{\displaystyle #1}{\displaystyle #2}}

\begin{document}
\maketitle


\textbf {
2. Considere  un  sistema  con  un  espacio  lógico  de  me
moria  de  128K  páginas (máximo espacio  de  memoria  virtual) 
con  8  KB  cada  una,  una  memoria  física  de  64  MB  y 
direccionamiento  al  nivel  de  byte.  ¿Cuántos  bits  hay  en  la  dirección  lógica?  ¿Y  en  la física? 
} \\




\end{document}